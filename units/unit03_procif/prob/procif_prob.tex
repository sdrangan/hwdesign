\documentclass[11pt]{article}

\usepackage{fullpage}
\usepackage{amsmath, amssymb, bm, cite, epsfig, psfrag}
\usepackage{graphicx}
\usepackage{float}
\usepackage{amsthm}
\usepackage{amsfonts}
\usepackage{cite}
\usepackage{hyperref}
\usepackage{pgf,tikz}
\usepackage{enumitem}
\usepackage{mathtools}
\usepackage{siunitx}
\usepackage{../../styles/course}


\begin{document}

\title{Introduction to Hardware Design\\
Memory and Basic Processor Interfaces}
\author{Profs. Sundeep Rangan, Siddharth Garg}
\date{}

\maketitle

\begin{enumerate}
\item \qtag{Sizing memory} \emph{Sizing memory}.  Suppose that a communications receiver
receives data at a rate of \SI{1}{Gsps} (giga sample per second) from
each antenna.  The receiver has 4 antennas and each sample is a complex
number represented by 12 bits for the real part and 12 bits for the
imaginary part (i.e., 24 bits per sample).

\begin{enumerate}[label=(\alph*)]
\item What is the total data rate in Gbps (giga bits per second) from
  all antennas?
\item Suppose there is a sample buffer of size \SI{8}{MB}.  How much time
    (in milliseconds) can the buffer hold before it is full?
\item Suppose that each memory can be written at a rate of \SI{400}{MHz}
with a bus width of 64 bits.  How many such memories are needed to handle
the data rate from part (a)?  Assume that you cannot split a
sample across multiple memories and you cannot split the real or imaginary
component of a sample ver multiple writes.
\end{enumerate}

\begin{solution}
Enter your solutions here.
\end{solution}

\item \qtag{FIR filter} \emph{FIR filter}:  
Consider implementing an FIR filter
or convolution.  An FIR filter takes an input signal $x[n]$, $n=0,\ldots,N-1$
 and produces
an output signal $y[n]$ via
\[
    y[n] = \sum_{k=0}^{K-1} h[k] x[n-k]
\]
where $h[0], \ldots, h[K-1]$ are the filter coefficients.  Suppose all values
are 16-bit signed integers.  Assume the arrays are length
$N=1000$ and the filter length is $K=32$.  

\begin{enumerate}[label=(\alph*)]
    \item What is the number of 32-bit words needed to store $x$, $y$ and $h$?
    \item Suppose that the filter coefficients are stored in registers,
    and each 32-bit word write takes 5 clock cycles to write.  How many 
    clock cycles are needed to program the filter coefficients?
    \item Suppose that FIR filter IP can run the filter operation for $N$
    length vectors in $N+K$ clock cycles.  How many clock cycles are needed to
    run the filter for length $N=1000$ vectors and $K=32$ filter coefficients?
    \item If the processing requires programming the filter coefficients,
    then running the filter, what percentage of the total time is spent
    programming the filter coefficients?
\end{enumerate}
\begin{solution}
Enter your solutions here.
\end{solution}


\item \qtag{AXI4-Lite Write} \emph{AXI4-Lite Write}:  Suppose 
a processor (master) writes to a hardware IP (slave) using the AXI4-Lite protocol.
Assume the following AXI4-Lite write timing:
\begin{itemize} 
\item On cycle 1, the master asserts \texttt{AWVALID} and \texttt{WVALID} 
with address on \texttt{AWADDR} and data on \texttt{WDATA}. 
\item On cycle 2, the slave asserts \texttt{AWREADY}. 
\item On cycle 3, the slave asserts \texttt{WREADY}. 
\item There is a 2-cycle delay from the time \emph{both} the address 
and data have been transferred to when the slave asserts \texttt{BVALID}. 
\item There is a 1-cycle delay from the time \texttt{BVALID} is asserted to 
when the master asserts \texttt{BREADY}. 
\end{itemize}
Answer the following questions based on the timing above:

\begin{enumerate}[label=(\alph*)] 
    \item On which cycle is the address transferred on the AW channel?
    \item On which cycle is the data transferred on the W channel? 
    \item On which cycle is \texttt{BVALID} asserted by the slave? 
    \item On which cycle is \texttt{BREADY} asserted by the master? 
    \item For each channel separately, on which cycle can the master first de-assert: 
    \texttt{AWVALID} and \texttt{WVALID}? 
    \item For each channel separately, on which cycle can the master first assert 
    \texttt{AWVALID} and \texttt{WVALID} for the \emph{next} write transaction? 
    If the master wants to start the next write with both \texttt{AWVALID} and 
    \texttt{WVALID} high in the same cycle, what is the earliest such cycle? 
\end{enumerate}

\begin{solution}
Enter your solutions here.
\end{solution}


% ------------------------------------------------------------
% AXI4-Lite Read Timing Problem (LaTeX Version)
% ------------------------------------------------------------

\item \qtag{AXI4-Lite Read} \emph{AXI4-Lite Read}:  
A processor (master) performs a read from a hardware IP (slave) using the AXI4-Lite protocol.  
The following timing behavior occurs:

\begin{itemize}
  \item Cycle 1: The master asserts \texttt{ARVALID} with the read address on \texttt{ARADDR}.
  \item Cycle 2: The slave asserts \texttt{ARREADY}.
  \item Cycle 3: The slave asserts \texttt{RVALID} with the read data on \texttt{RDATA}.
  \item Cycle 4: The master asserts \texttt{RREADY}.
  \item The slave has a 1-cycle latency from the time the address is transferred to the time it asserts \texttt{RVALID}.
  \item The master has a 1-cycle delay from the time \texttt{RVALID} is asserted to the time it asserts \texttt{RREADY}.
\end{itemize}

Answer the following:

\begin{enumerate}[label=(\alph*)]
  \item On which cycle is the read address transferred?
  \item On which cycle is the read data transferred?
  \item On which cycle may the master de-assert \texttt{ARVALID}?
  \item On which cycle may the slave de-assert \texttt{RVALID}?
  \item On which cycle may the master begin the next read transaction (i.e., assert \texttt{ARVALID} for the next read)?
\end{enumerate}

\begin{solution}
Enter your solutions here.
\end{solution}


\item \qtag{IP interface}  \emph{IP interface}:  
Suppose an IP implements the iterative update:
\[
    x[n+1] = f(x[n], a, b)
\]
where $f(x,a,b)$ is some known function, and $a$ and $b$ are constants.
The IP is given an initial value $x[0]$ and returns the final value $x[N]$
after a specified number of iterations $N$.
Assume all values are 32-bit signed integers.

\begin{enumerate}[label=(\alph*)]
    \item You decide to write a Vitis HLS function to implement this IP.

\begin{ccode}
void diff_eq_solver(...)

#pragma HLS INTERFACE s_axilite port=... bundle=CTRL_BUS
// ...
\end{ccode}

What arguments would you include in the function, and what \#pragma HLS INTERFACE
lines would you write?  You do \emph{not} need to write the body of the function.

    \item Vitis HLS will automatically build the AXI4-Lite interface.
Assuming it adds a 32-bit \texttt{AP\_CTRL} register, what is the register
map for this IP?  Assume the base address is 0x00 and each register is
32 bits wide.
\end{enumerate}

\begin{solution}
Enter your solutions here.
\end{solution}


\end{enumerate}


  \end{document}

